\documentclass[10pt,a4paper]{article}
\usepackage[utf8]{inputenc}
\usepackage[T1]{fontenc}
\usepackage{amsmath}
\usepackage{amssymb}
\usepackage{graphicx}
\usepackage[russian]{babel}
\title{Отчет по лабораторной работе №1}
\author{Ивлев А.Е Б19-511}
\begin{document}
	\maketitle
	
	Исходная динамическая система:
	
	\begin{equation}
		\label{math/1}
		\ddot{x} + \mu \dot{x} + \alpha (\exp(x) - 1) = 0.	
	\end{equation}
	
	Она же в каноническом виде:
	\begin{equation}
		\label{math/2}
		\begin{cases}
			\dot{x} = y, \\
			\dot{y} = -\mu y - \alpha (\exp(x) - 1).
		\end{cases}
	\end{equation}

	Приравнивая правые части уравнения к нулю, находим точки покоя:
	\begin{equation}
		\label{math/3}
		\begin{cases}
			\ y = 0, \\
			\ -\mu y - \alpha (\exp(x) - 1) = 0.
		\end{cases}
	\end{equation}
	
	Единственная точка покоя - точка $(0;0)$.
	
	Матрица Якоби системы (\ref{math/1}) в точке покоя:
	\begin{equation}
		\label{math/4}
		\begin{bmatrix}
			0& 1\\
			-\alpha& -\mu
		\end{bmatrix}
	\end{equation}
	
	Собственные значения матрицы Якоби:
	\begin{equation}
		\label{math/5}
		\lambda_{1,2} = \frac{-\mu \pm \sqrt{\mu^{2} - 4\alpha}}{2}
	\end{equation}
	
	
\end{document}